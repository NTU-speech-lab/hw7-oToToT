\documentclass[a4paper,11pt]{article}
\usepackage[left=1.2cm, right=1.5cm, top=2cm, bottom=2cm]{geometry}
\usepackage{xeCJK}
\usepackage{indentfirst}
\usepackage{tabularx}
\usepackage{graphicx}
\usepackage{amsmath}
\usepackage{listings}
\usepackage{verbatim}
\usepackage{fancyhdr}
% \usepackage[compact]{titlesec}
\usepackage[usenames,dvipsnames]{xcolor}

\setCJKmainfont{NotoSansCJKtc-Thin}
\setmonofont{Consolas}

\definecolor{CodeGreen}{rgb}{0,0.6,0}
\definecolor{CodeGray}{rgb}{0.5,0.5,0.5}
\definecolor{CodeMauve}{rgb}{0.58,0,0.82}
\lstset{
    basicstyle = \ttfamily\footnotesize, 
    breakatwhitespace = false,
    breaklines = true,         
    commentstyle = \color{CodeGreen}\bfseries,
    extendedchars = false,
    keepspaces=true,
    keywordstyle=\color{blue}\bfseries, % keyword style
    language = C++,                     % the language of code
    otherkeywords={string},
    numbers=left,
    numbersep=5pt,
    numberstyle=\tiny\color{CodeGray},
    rulecolor=\color{black},
    showspaces=false,
    showstringspaces=false,
    showtabs=false,
    stepnumber=1,       
    stringstyle=\color{CodeMauve},        % string literal style
    tabsize=2,
}
% from https://blog.csdn.net/RobertChenGuangzhi/article/details/45126785

\title{Machine Learning 2020 - Homework 7 Report}
\author{學號:b08902100, 系級:資工一, 姓名:江昱勳}
\date{}

\begin{document}
\pagestyle{fancy}
\fancyhead[L]{Machine Learning 2020 - Homework 7}
\fancyhead[R]{Author: b08902100 江昱勳}

\maketitle

% \verbatiminput{HW2_S.txt}
% \lstinputlisting{HW2.cpp}

\begin{enumerate}
\item 請從 Network Pruning/Quantization/Knowledge Distillation/Low Rank Approximation/Design Architecture  選擇兩者實做並詳述你的方法,將同一個大 model 壓縮至接近相同的參數量,並紀錄其 accuracy。

這裡我搭建了兩個模型。

第一個模型是想實驗模型設計的部份(把模型變瘦但變深),總共有13層的CNN並接上3層的全連接層,每一層CNN層都是由一組Depthwise Convolution + Pointwise Convolution構成,前5層的活化函數皆採用ReLU6,後面8層則採用swish,而僅只有前4層有使用max pooling,channel的部份分別為16,32,32,64,64,64,64,128,128,128,128,128,128,128,最後全連接層的部分則為160,160,11,總共有186123個參數,不過此model實際訓練的時候似乎訓練不起來,有可能是過深的原因,訓練了200個epoch後在validation上的accuracy仍只有0.205539,與其在training set上的accuracy差不多。

第二個模型主要是嘗試pruning的部份,原本的模型總共有8層CNN並接上3層的全連接層,每一層CNN層都是由一組Depthwise Convolution + Pointwise Convolution構成,前5層的活化函數皆採用ReLU6,後面3層則採用swish,而僅只有前4層有使用max pooling,channel的部份分別為24,48,96,96,192,192,192,192,最後全連接層的部分則為160,160,11,該模型在validation上的正確率為0.846064。
使用0.95的pruning rate後參數減少為186750,而validation的正確率也減少至0.834694。

\item {[Knowledge Distillation]} 請嘗試比較以下 validation accuracy (兩個 Teacher Net 由助教提供)以及 student 的總參數量以及架構,並嘗試解釋為甚麼有這樣的結果。你的 Student Net 的參數量必須要小於 Teacher Net 的參數量。

\begin{enumerate}
    \item[x.] Teacher net architecture and \# of parameters: torchvision's ResNet18, with 11182155 parameters.
    \item[y.] Student net architecture and \# of parameters: 144389
    \item[a.] Teacher net (ResNet18) from scratch: 80.09\%
    \item[b.] Teacher net (ResNet18) ImageNet pretrained \& fine-tune: 88.41\%
    \item[c.] Your student net from scratch: 78.8630 \%
    \item[d.] Your student net KD from (a.): 79.1254 \%
    \item[e.] Your student net KD from (b.): 80.6414 \%
\end{enumerate}

我的student net模型總共有9層CNN並接上3層的全連接層,每一層CNN層都是由一組Depthwise Convolution + Pointwise Convolution構成,前5層的活化函數皆採用ReLU6,後面4層則採用swish,而僅只有前4層有使用max pooling,channel的部份分別為19,38,76,76,76,152,152,152,152,最後全連接層的部分則為128,128,11。

以上訓練結果皆為訓練200 epoch後的結果,可以發現原先train from scratch即可達到78\%左右的準確率,而因此讓他跟80\%的模型學習則可以讓他在200 epoch時就幾乎將teacher學會的東西學走達到79\%的準確率,而若是讓他與更加厲害的老師學習則可以看到在200 epoch時就能學到80\%的準確率,這甚至超過了train from scratch的ResNet18,因此可以看到老師的準確率會影響模型學習的成果,如果與準確率高的模型學習可以較快達到一定的準確率。

\item {[Network Pruning]} 請使用兩種以上的 pruning rate 畫出 X 軸為參數量,Y 軸為 validation accuracy 的折線圖。你的圖上應該會有兩條以上的折線。

我分別使用了0.95的pruning rate與0.85的pruning rate做實驗,以下是參數量對validation accuracy的折線圖。

\includegraphics[width=.85\textwidth]{arg.pdf}

此外我也把訓練過程記錄下來了,我這裡是以每200個epoch就prune一次,以下是訓練時的validation accuracy與training accuracy,以及loss的部份。

\includegraphics[width=.85\textwidth]{acc.pdf}

\includegraphics[width=.85\textwidth]{loss.pdf}


\end{enumerate}

\end{document}
